% xelatex
\documentclass{article}

\usepackage[utf8]{inputenc}
\usepackage{fontspec}
\usepackage[margin=.75in]{geometry}
\usepackage{graphicx}
\graphicspath{{./assets/}}
\usepackage{ot-tableau}
\usepackage[backend=biber, style=authoryear-icomp]{biblatex}
\usepackage{textgreek}
\usepackage{easylist}
\usepackage{hanging}
\usepackage{hyperref}
\usepackage{blindtext}
\usepackage{tipa}
\usepackage{cgloss4e}
\usepackage{gb4e}
\usepackage{qtree}
\usepackage{enumerate}
\usepackage{longtable}
\usepackage{textgreek}
\usepackage{amsmath,amssymb,latexsym}
\usepackage{wasysym}
\setlength{\parindent}{0cm}

\pagenumbering{roman}

\title{Ancient History Final Review}
\author{Christopher Milan}

\begin{document}

\maketitle
\setcounter{section}{-1}
\section{General information}
\subsection{Format}
\subsubsection{Short answers (2 questions of 15 points each)}
You will be asked to answer two questions on topics covered in these units.  To prepare
for this section,  review your homework and class notes for all discussion questions. (The
exam questions will not be identical to discussion questions, but they will cover the
same material. In this section you will not be asked to discuss specific primary sources,
but knowledge derived from primary sources may still be relevant to your answers.) You
will write a paragraph addressing each question. In this section, you will not be asked to
provide dates, but you will need to know the sequence of events.
\subsubsection{Passage analyses of previously covered sources  (2 questions of 25 points each)}
qYou will identify and analyze two passages from the primary sources we read in these
units. The format will be very similar to the passage analyses on the midyear exam and
on the Unit 7 test. For each passage, you will answer short identification questions based
on the primary source charts and write a paragraph answering an analytical question.
Note that the short questions may include dates. (It is always fine to give an approximate
date, such as mid-2nd century BCE.)
\subsubsection{Passage analysis of an unseen source (1 question - 20 points)}
   For the unseen passage, you will write a paragraph answering an analytical question.
   The format will be similar to the unseen passage on the midyear exam
\subsection{Review process}
  \subsubsection{Before you start your review, collect all materials and organize them by unit}
  Check for any that are missing, especially class notes from days you were absent.
  Check Drive to be sure that you use all unit materials and Toolbox instruction sheets. \\
  \textbf{Note} This should all be here.
  \subsubsection{Use what you know about the exam format to help you structure your review}
  Give yourself review tasks that match what you will be asked to do on the exam.
  \subsubsection{Make conscious decisions about your time allocation; don’t let it just happen}
  Start early, and take breaks when you feel yourself becoming less productive.
  Remember that multiple review sessions will give you better retention than one
  mammoth session. (Memory researchers have said that studying the same material in
  different places - for instance, different rooms in your house - will also boost your
  retention.) Estimate how much time you need for a given task, such as reviewing the
  discussion questions for a single assignment. Test your estimates as you go, and
  reallocate your time as needed.
  \subsubsection{Keep your review active, and quiz yourself as you go}
  Do not simply reread or “go over” your notes.  Passive review strategies prepare you to
  recognize the correct answer (as you would on a multiple choice test), not to produce it.
  For more on the benefits of active review strategies, look at the Toolbox handouts by
  Daniel Willingham and Benedict Carey. \\
  \textbf{Note} This means don't just study off this guide.
  \subsubsection{Remember that all review materials are up on Drive in the “Final Review” folder}
  \subsubsection{Review procedures}
    \subparagraph{Review your discussion question notes}
    As a first step, consolidate your homework and class notes for each discussion
    question. (This is especially important if you did not take class notes on the same page
    as homework notes). You may wish to download unit assignment sheets and type in
    your notes. \\
    Once you have done that, organize your notes as thoughtfully as you can. Remember
    that for review bullet points tend to work better than sentences. What are your main
    points/key phrases? What are your supporting details? What is the relation between
    different questions for the same assignment? How about different assignments in the
    same unit? Return to the textbook for clarification if you need to, but do not simply
    reread assignments.  (That’s likely to lead you into the “fluency illusion”!) \\
    \textbf{Quiz yourself}
    For this step, many students find flashcards useful. Put the question on
    one side and the main points of your answer on the other side. See how many of your
    key points you can remember and how you would explain their relation to each other.
    You are more likely to remember material you reviewed thoughtfully.  (As Daniel
    Willingham says: “Memory is the residue of thought.”) \\
    \textbf{Note}
    This is all very helpful as supplementary review for this guide.
    \paragraph{Review your primary source charts and marginal notes}
    For the passages, follow the instructions on the “Reviewing Primary Sources” handout
    in Toolbox. Start with your primary source charts. Complete any that are missing, and
    give some thought to how you would identify each source and avoid confusing it with
    any similar sources. Be sure that as you review you look back at the unit assignment
    sheets as well as your class notes. Often the discussion questions will give you key
    terms as well as reminding you why we read this source and what was important about
    it. Finally, remember that passage analysis questions ask you to use both the text itself
    and your knowledge of its context. Think about which textbook or secondary source
    assignments provide meaningful context for each text.
    \paragraph{A note on group study}
    Many students enjoy doing some of their studying with a friend or a small group. This
    can work well as long as you do it right. Most important: be sure that each of you
    studies the material individually before you get together to quiz each other.
  \subsubsection{And finally...}
  It’s easiest to review by sections, but on the exam itself you will want to put all your
  knowledge together.  For instance, even if a passage you reviewed does not appear on
  the exam, your memory of it may help you with a short answer or an unseen passage.
  Look back at the Toolbox handout “Writing Essays on Tests and Exams” for more tips
  about what to do during the exam itself.  \\
  Ms. Budding will accept emailed review questions through Saturday June 2. Ms. Budding
  will answer these questions – most often by telling you which readings to look at - in a
  group email Sunday morning.  That’s it. Give the review your best shot, but don’t get too
  worried about the whole enterprise.   And don’t go short on sleep: being alert is likely to
  help you more than knowing a little more material.  Good luck!
\section{Primary Souce Review}
\subsection*{Arrian, \textit{Alexander the Great}}
\subsubsection*{Primary Source Chart}
There was no primary source chart distributed for Arrian's \textit{Alexander the Great}.
\subsubsection*{Assignment 6A.6}
\begin{enumerate}
  \item Why did Alexander have Bessus executed, and why did he have him mutilated first? \\
  Alexander justified his mutilation of Bessus as punishment for his betrayal of Darius. However
  it is more likely that Alexander want to appear to his Persian subjects as a legitimate king.
  \item Why does Arrian criticize Alexander’s treatment of Bessus? (Consider both Arrian’s
  background and perspective, and Alexander’s actions.) \\
  Arrian believed that Alexander's treatment was barbaric (Arrian being Roman.)
  \item What were the reasons for Cleitus’s death? What does it suggest about relations
  between Alexander and his court at this time? \\
  Clietus, drunk, was killed by Alexander for insulting him.
\end{enumerate}
\subsubsection*{Assignment 6B.3}
\begin{enumerate}
  \item What significance did the custom of proskynesis have for Persians? \\
  Proskynesis was a display of respect, usually shown to the king. It was required for Persian
  subjects to approach the Great King.
  \item How did the Greeks and Macedonians interpret proskynesis? (Consider the
  arguments made by Callisthenes.) \\
  The Greeks thought that worshipping a king in such a fashion was disrespectful to the gods.
  Some Greeks refused to perform the ritual, even under the pain of death. There was also a
  fear that people who did npt deserve this treatment would still demand it, which would be
  even more disrespectful to the gods.
  \item What can we surmise about Alexander’s motives for attempting to introduce proskynesis? \\
  Alexander likely wanted to appear as a legitimate king to the Persians.
  \item According to Arrian, why did Alexander choose to march across the Gedrosia desert?
  What hardships did the troops encounter in crossing it? \\
  Cyrus and Semiramis had already failed to cross it, and Alexander wanted to be comperable,
  or better than they were. Much of his army was killed, and the pack animals died. Many people
  were left behind, or washed away by the monsoons. They also died from drinking too much
  water upon encountering a water source.
  \item What character traits did Alexander show during the march? How did the march
  affect his relationship with his soldiers, as far as you can tell? \\
  Alexander led the way for his troops, making it easier for them to walk through the sand.
  He also refused special treatment when water was found.
\end{enumerate}
\subsubsection*{Assignment 6B.4}
\begin{enumerate}
  \item Who was involved in the Susa wedding (as bride grooms and as brides)? What do
  you think Alexander hoped to achieve through the Susa weddings? \\
  Alexander and his companions were maried to Darius' daughers ant other Persian royalty.
  Alexander hoped that this intermarriage would bring the Persians and Greeks closer.
  \item Why did the soldiers mutiny at Opis? Considering both the soldiers’ actions and
  Alexander’s response, how would you compare this mutiny with the Hyphasis mutiny?
  Focus on Alexander’s relations with his soldiers. \\
  The Greek soldiers revolted at Opis because of Alexander's Median attire, the introduction
  of epigone (young Persian fighters) into the army, the Persian weddings in Susa, and the 5th
  cavalry regiment, which was "barbarian", but still used Macedonian equipment. The soldiers
  believed that Alexander himself was becoming barbaric. Similarly to in the Hyphasis mutiny,
  Alexander shamed his soldiers, explained all his achievements, and used their emotions
  to make them feel bad. He again secluded himself for three days.
\end{enumerate}
\subsubsection*{Assignment 6B.6}
\begin{enumerate}
  \item Some biographers have seen in Alexander’s actions at this period proof of mental
  imbalance (paranoia or megalomania). How would you use the text to support or counter this view? \\
Alexander is likely not a megalomaniac, because he clearly held Haphaestion at a very high value
he seemed very distraught over his death, and some sources say he asked the oracle if he could
worship Haphaestion as a god. He didn't replace his position in the army, and ordered the construction
of temples and shrines honouring Haphaestion in Egypt. Also, while on his deathbed, Alexander
made a point to personaly greet all of the soldiers who came to meet him.
  \item What does Arrian’s evaluation of Alexander tell us about Alexander? \\
  Alexander lived until age 32, and reigned for 12 years. He was great at raising morale, never
  broke promises, and didn't hold grudges. He felt bad for his misdeeds and was greatly loved
  by his soldiers.
  \item What does Arrian’s evaluation of Alexander tell us about Arrian? \\
  Arrian looks over Alexander's faults to admire him. Arrian admires Alexander's capability of
  remorse, and shows his stoicism in his praise of Alexander. Although Arrian disliked Alexander's
  claims of godlike heritage, his Persian attire, and prolonged "carousals", he asks critics what
  they would have done in his place.
\end{enumerate}
\subsection*{Unit 7 (The Hellenistic Age)}
\subsubsection*{Letter to Zenon}
\textbf{Genre}
Private Letter \\
\textbf{Location of Source}
7.2, 7 Handouts Hellenistic \\
\textbf{Content}
A non-Greek (perhaps Arab) worker who took care of camels in Ptolemaic Egypt complains
that he has been treated unfairly because he cannot “act the Hellene” or perhaps “speak
Greek.”  (The original word could mean one or both.)  Note that the people referred to in the
text (Zenon himself, Krotos, and Jason) all have Greek names. \\
\textbf{Dates}
c. 256 BCE (for events and source) \\
\textbf{Context}
In the Hellenistic world, including Ptolemaic Egypt, knowledge of Greek language and
customs became crucial. \\
\textbf{Author}
We know that he was of low social status (caring for camels) and that he was not Hellenized. \\
\textbf{Purpose}
It was written to induce the author’s employer, the official Zenon, to redress the wrongs
done to the author by Zenon’s subordinates. \\
\\
\textbf{Discussion questions}
\begin{enumerate}
  \item Using these sources, what can you deduce about relations between Greeks (or
  Greek-speakers) and non-Greeks in Ptolemaic Egypt? Do the sources present a consistent picture? \\
  Greeks thought of non-Greeks as barbarians, and often disrespected them. Non-Greeks in
  general had more subservient jobs. Non-Greeks were also forced to follow Greek customs.
  \item How did the inhabitants of Ptolemaic Egypt act on their grievances? How could an
  individual obtain protection in this society? \\
  Inhabitants of Egypt would send letters to their superiors to complain. They would also tell
  the policing system. The king was considered to be the benefactor for everyone.
\end{enumerate}
\subsubsection*{Scalding in the Baths}
\textbf{Genre}
Letter/Petition \\
\textbf{Location of Source}
7.2, 7 Handouts Hellenistic \\
\textbf{Content}
A Greek female inhabitant of Ptolemaic Egypt complains that an Egyptian bath attendant
poured overly hot water on her, scalding her. \\
\textbf{Dates}
c. 221 BCE (for events and source) \\
\textbf{Context}
In the Hellenistic world, including Ptolemaic Egypt, knowledge of Greek language and
customs became crucial. \\
\textbf{Author}
We know that the author was ethnically Greek. \\
\textbf{Purpose}
It was written to induce the king’s officials to send the bath man for trial. \\
\\
\textbf{Discussion questions}
\begin{enumerate}
  \item Using these sources, what can you deduce about relations between Greeks (or
  Greek-speakers) and non-Greeks in Ptolemaic Egypt? Do the sources present a consistent picture? \\
  Greeks thought of non-Greeks as barbarians, and often disrespected them. Non-Greeks in
  general had more subservient jobs. Non-Greeks were also forced to follow Greek customs.
  \item How did the inhabitants of Ptolemaic Egypt act on their grievances? How could an
  individual obtain protection in this society? \\
  Inhabitants of Egypt would send letters to their superiors to complain. They would also tell
  the policing system. The king was considered to be the benefactor for everyone.
\end{enumerate}
\subsubsection*{Decree of Antiochus III concerning the Jews}
\textbf{Genre}
Official decree \\
\textbf{Location of Source}
7.3, 7 Handouts Hellenistic \\
\textbf{Content}
Antiochus III announces privileges that he will grant to the Jews of Judea. \\
\textbf{Dates}
c. 200 BCE (for events and source) \\
\textbf{Context}
Judea had just passed from Ptolemaic to Seleucid control as the result of a war between
the two kingdoms.  The Ptolemies and Seleucids were frequently at war, with the Levant
as a frontier zone. \\
\textbf{Author}
The author was Antiochus III, who ruled 223-187. \\
\textbf{Purpose}
Antiochus III likely followed this path with the Jews in order to gain their trust and avoid a
revolt. \\
\\
\textbf{Discussion questions}
\begin{enumerate}
  \item Antiochus III’s Decree concerning the Jews grants the Jewish community a number
  of privileges. What were some of the more important ones? What do you think the king’s
  motive was? \\
  Antiochus III offered material gifts for sacrifice, materials for a temple (which were very specific),
  tax exemptions, and no new laws to be forced upon the people. Antiochus III wanted the
  Jews to be loyal, he wanted to reward them, and he wanted them to recover.
\end{enumerate}
\subsubsection*{Livy on Antiochus IV}
\textbf{Genre}
History \\
\textbf{Location of Source}
7.3, 7 Handouts Hellenistic \\
\textbf{Content}
Livy lists Antiochus IV’s “gifts to cities and worship of gods." \\
\textbf{Dates}
The events took place during 175-64 BCE (the reign of Antiochus IV).  The source was
written in the late first century BCE./first century CE. \\
\textbf{Context}
\begin{itemize}
  \item Rome’s power in the eastern Mediterranean was growing in the mid-2nd century,
  and Antiochus had been a hostage in Rome.
  \item Giving major gifts to cities and building temples were typical activities for
  Hellenistic rulers.
\end{itemize}
\textbf{Author}
The author was Livy,  a Roman historian who lived 59 BCE to 17 CE. He wrote during the
reign of Augustus, and told the story of Rome’s rise. \\
\textbf{Purpose}
Livy glorifies Rome. \\
\\
\textbf{Discussion questions}
\begin{enumerate}
  \item Using only Livy, what impression would you get of Antiochus IV? \\
  Judging from Livy, Antiochus IV was a good, generous, and kind king. He gave gifts to
  everyday people, and admired Roman customs.
\end{enumerate}
\subsubsection*{Polybius on Antiochus IV}
\textbf{Genre}
History \\
\textbf{Location of Source}
7.3, 7 Handouts Hellenistic \\
\textbf{Content}
In the first paragraph, Polybius describes various unusual (for a king) actions of Antiochus IV. \\
In the second paragraph, he describes how a Roman magistrate, Popilius, forced Antiochus
to end his war with Ptolemy and withdraw his troops from Egypt. \\
\textbf{Dates}
The events took place during 175-64 BCE (the reign of Antiochus IV), with the intervention
of Popilius happening in 168 BCE. The source is roughly contemporary.  \\
\textbf{Context}
Rome’s power in the eastern Mediterranean was growing in the mid-2nd century, and
Antiochus had been a hostage in Rome. \\
\textbf{Author}
The author was Polybius (203 BCE to 120 BCE), a Greek taken to Rome as a hostage.
He became the foremost historian of the Hellenistic age, admired Roman institutions, and
wrote with the aim of explaining Rome’s rise. \\
\textbf{Purpose}
Polybius glorifies Rome. \\
\\
\textbf{Discussion questions}
\begin{enumerate}
  \item What do the selections from Polybius suggest about Rome’s position in the
  Mediterranean world at this time, and about Antiochus IV’s relationship with and attitudes
  toward Rome and Romans? \\
  Rome was clearly very powerful, and arrogant with their power. Antiochus wanted to be friendly
  with Rome, but was greatly disrespected and humiliated by their correspondant.
\end{enumerate}
\subsubsection*{First Maccabees, Chapters 1 and 2}
\textbf{Genre}
Scripture (religious text). \\
\textbf{Location of Source}
7.4, 7 Handouts Hellenistic \\
\textbf{Content}
The author recounts Antiochus IV hellenization of Judea, whether voluntary or involuntary.
This includes disrespecting Judaism by forcing the inhabitants to conform to a pagan religion.
Then the story of Mattathias and his sons is recounted, where they revolt against Antiochus IV.
Mattathias also kills a Jew who was conforming to Antiochus IV's decrees. \\
\textbf{Dates}
\begin{itemize}
  \item First Maccabees as a whole describes events between 166 (when the revolt began)
  and 135 BCE.
  \item Historians have deduced that it was written between 135 and 63 BCE.  It must have
  been written after 135 BCE (because it mentions events of that date), but before 63 BCE,
  when the Romans conquered Judea (because the Romans are described favorably).
\end{itemize}
\textbf{Context}
At the time of the revolt, Judea was part of the Seleucid kingdom ruled by Antiochus IV
(r. 175-64). The revolt was ultimately successful:  under the Hasmoneans (descendants
of the Maccabees), Judea won autonomy from the Seleucids in 147 BCE. \\
\textbf{Author}
We have no external knowledge about this author. From evidence internal to the source, scholars say:
\begin{enumerate}
  \item the source was probably originally written in Hebrew (shown by Hebrew idioms),
  though preserved only in Greek translation.
  \item The author was probably a Jew living in Judea. (He knows the geography of Palestine
  well, but knows little of foreign countries.)  The author admires the Hasmonean dynasty
  (who were descendants of Judah Maccabeus).  Some have speculated that he was a
  Hasmonean court historian.
\end{enumerate}
\textbf{Purpose}
The source was written for an audience of Jews living in Judea (that is, not in the diaspora).
It emphasizes that God chose the Hasmonean family to save Israel, and it may have been
written to glorify the Hasmonean kings. \\
\\
\textbf{Discussion questions}
\begin{enumerate}
  \item First Maccabees describes Antiochus IV’s attempt to repress the practice of Judaism
  in Judea. This represented a drastic departure from general Seleucid religious policy, and
  specifically from the policies pursued by Antiochus IV’s father, Antiochus III. What are
  some possible reasons for Antiochus IV’s actions? \\
  Antiochus wanted to show his power, after being disrespected by Rome. He also wanted a
  base of operations against Ptolemy. It is unknown why exactly he acted in the way, and it is
  possible he was insane.
  \item According to First Maccabees, what were some of the features of Hellenization in Judea? \\
  The volunary aspect of Hellenization was the addition of a gymnasium. However, everything
  else, being new customs, a bigger military prescense, sacrifice of unclean animals, and different
  worship, was involuntary.
  \item Who supported Hellenization, and why? \\
  The Gentiles suported Antiochus IV and Hellenization.
  \item According to First Maccabees, whom did Mattathias and his sons attack, and why? \\
  Mattathias himself attacked a Jew, who had submitted to the new customs imposed on the
  people of Judea. Then he and his sons fleed. They were attacked, and chose to retaliate by
  attacking sinners.
\end{enumerate}
\subsection*{Unit 8 (Roman Society)}
\subsubsection*{Inscription from Puteoli/L’annee epigraphique}
\textbf{Genre}
Official regulations \\
\textbf{Location of Source}
8.4, \textit{The Roman World}, 26-27. \\
\textbf{Content}
[from the editor’s introduction]: “It describes the regulations that governed the conduct of
funerals, which were normally carried out by professional undertakers, who could also be
hired to torture or execute slaves, either by order of a court or at their owner’s request.” \\
\textbf{Dates}
Probably the first century CE. \\
\textbf{Context}
Rome had an abundant supply of slaves from their rapidly expanding empire. This allowed
extreme brutality towards those slaves, as there would always be more to replace them.
However, slaves becomming free was very common. \\
\textbf{Author}
Municipal government of Puteoli. \\
\textbf{Purpose}
To establish regulations about how punishments should be carried out, whether they are
flogging, hanging, etc. \\
\textbf{Quote} \\
"Then to the contractor or to his partner, as often as anyone shall throw out [a corpse unburied(?)],
he shall pay a fine of sixty sesterces per body, and the magistrate shall enforce judgement
for recovery of this sum in accordance with the law of the colony."
\\
\textbf{Discussion questions}
\begin{enumerate}
  \item Using this source, what can you deduce about the treatment of slaves in Roman society? \\
  There was extreme brutality towards Roman slaves, and no limits on the owner's power.
  The slaves came from the land Rome conquered, and were the prisoners of war. However,
  slaves becoming free was very common.
  \item What else can you deduce? \\
  Slaves had different statuses, depending on whether they were educated, whether they worked
  in agriculture, and whether they were expensive or cheap.
\end{enumerate}
\subsubsection*{The Twelve Tables}
\textbf{Genre}
Laws \\
\textbf{Location of Source}
8.5, \textit{The Roman World}, 4-9 \\
\textbf{Content}
The Twelve Tables consist of laws from the early republican period.  Note that the tables
themselves have not survived, even in a damaged form.  Scholars have attempted to
reproduce them by using parts of laws that were quoted in later cases.  Many parts are
missing, and sometimes the meaning and/or correct order of fragments is unclear. \\
\textbf{Dates}
According to later Roman tradition, the Twelve Tables were written down around 450 BCE. \\
\textbf{Context}
According to later Roman tradition, the Twelve Tables were written down as a result of the
“Struggle of the Orders” (conflict between patricians and plebeians). \\
\textbf{Author}
According to later Roman tradition, ten patrician men were appointed to write down the laws. \\
\textbf{Purpose}
The laws were written to regulate conflicts typical of the early republic (and, more broadly,
typical of many agrarian societies). \\
\\
\textbf{Discussion questions}
\begin{enumerate}
  \item Using the Twelve Tables, what can you deduce about the powers and limitations
  of the Roman state?  (What actions did individuals perform that a government agency
  would perform today?) \\
  The plaintiff was responsible for ensuring that the defendant came to court and that they were telling
  the truth. Theft at night could be punished immediately by death, and fathers had complete
  control over the life (or death) of their children.
  \item What can you deduce about traditional Roman values and beliefs?
\end{enumerate}
\subsubsection*{Plutarch, Life of Cato the Elder}
\textbf{Genre}
Biography \\
\textbf{Location of Source}
8.6, \textit{The Roman World}, 9-13 \\
\textbf{Content}
This gives an overview of Cato's life, and his values and beliefs. These include his opinions
on slaves, foreigners, how should family work, and other matters.\\
\textbf{Dates}
Written in about 100 CE, about events from 234 BCE - 149 BCE \\
\textbf{Context}
Rome expanded rapidly during Cato’s lifetime, bringing more exposure to foreigners and
foreign customs. \\
\textbf{Author}
Plutarch, the author was a priest and sometimes a lecturer, who wrote as a historian. He was
ethnically Greek, but recieved Roman citizenship. \\
\textbf{Purpose}
Plutarch used Cato to illustrate old Roman virtues. \\
\\
\textbf{Discussion questions}
\begin{enumerate}
  \item Using this source, make a list of traditional Roman virtues. Indicate the passage
  that supports each point. \\
  One should not be overly extravangant, even if they are rich. Men should be good husbands
  and fathers, and capable managers of slaves. Stoicism was key. Slaves should not be teachers
  of children, and their fathers ought to teach them instead.
  \item Using this source, make a list of things an “old Roman” like Cato might view with
  suspicion. Again, indicate the passage that supports each point. \\
  The philosophy of the Greeks and Greek culture in general was viewed with suspicion, along
  with slaves speaking with one another, stemming from the fear that they were plotting against
  their masters.
  \item Using this source, what can you add to your previous deductions about the treatment
  of slaves? \\
  Slaves should be immediately sold if they are not useful. There was a great fear of them
  revolting, and they were usually prisoners of war. Masters also required that  their slaves
  were perfect, and the imperfect slaves would recieve beatings.
\end{enumerate}
\subsection*{Unit 9 (The Late Republic)}
\subsubsection*{Handbook on Campaigning for Office}
\textbf{Genre}
Non-fiction; manual of advice. \\
\textbf{Location of Source}
9.3 \textit{Roman World}, 107-114 \\
\textbf{Content}
Quintus writes to his brother Marcus about how to campaign for office. Quintus explains that
because Marcus is a new man, he will need to be very precise about how he campaigns.
Quintus emphasizes that Marcus should make a great deal of friendships and allies. \\
\textbf{Dates}
The source was written to be used in Cicero’s campaign for the consulship in 63 BCE. \\
\textbf{Context}
The republic was unstable, with optimates and populares competing politically, and warlords
accumulating power. Cicero presented himself as an optimatus (defending the interests of
the Senate and the “best” people). \\
\textbf{Author}
This is usually attributed to Quintus Cicero, but it could be forged. \\
\textbf{Purpose}
Written by Quintus Cicero to help his brother’s political campaign. \\
\\
\textbf{Discussion questions}
\begin{enumerate}
  \item As the editor explains, in Roman politics the term “new man” was applied to someone
  who was the first in his family to become a consul. According to Quintus, how should Cicero
  take his “new man” status into account in campaigning for the consulship? \\
  Cicero needs to make sure his actions are perfect, as him being a "new man" means that
  his every action will be judged more heavily than others would be. Quintus emphasizes
  Cicero will need to make many connections with almost everyone, but he should prioritize
  the more important people.
  \item What methods does Quintus recommend for mobilizing support? \\
  Quintus tells Cicero to make sure he has enough "friends", and to make a stragey to
  become friends with the entire city. He tells Cicero that men will vote for him for one of
  these three things:
  \begin{enumerate}
    \item A kindness they recieve from him
    \item Hope that he will be a good leader
    \item Like mindedness or affection
  \end{enumerate}
  \item What is the point of having a “large crowd” accompany the candidate to the Forum
  (p. 114)? \\
  A big crowd allows Cicero to make a good impression on people, and show off his
  supporters.
\end{enumerate}
\subsection*{Unit 10 (Origins of Christianity)}
\subsubsection*{Letters between Pliny and Trajan}
\textbf{Genre}
Official letters \\
\textbf{Location of Source}
10.2, \textit{The Roman World}, 224-26 \\
\textbf{Content}
Pliny writes a letter to Trajan, asking about how Christians ought to be treated. There are a
few specific questions he asks, pertaining to whether they admit to being Christian, and if so,
when the do so. Trajan emphasizes in his response that it is unnecessary for Christians to be
hunted out, but they should be reprocussed unless then repent.\\
\textbf{Dates}
Around 111/112 CE (for both events and letters) \\
\textbf{Context}
Rome is very wary of Christianity at this time, as it is the worship of Christ, a Roman outlaw.
They also fear the nighttime meetings, and some suspect that cannibalism is involved in
Christianity. \\
\textbf{Author}
From the editor’s notes on Pliny (for an earlier source): “Gaius Plinius Caecilius Secundus,
nephew and adopted son of Pliny the Elder...was born at Comum (now Como), in northern
Italy, about AD 61.  He had a long and distinguished career as a lawyer and public servant,
culminating in his appointment as governor of the province of Bithynia-Pontus (in what is
now northern Turkey) in about 111-12” (Roman World, p. 58). \\
\textbf{Purpose}
Pliny wrote this letter in order to recieve guidance from his superior, Trajan. \\
\\
\textbf{Discussion questions}
\begin{enumerate}
  \item How did the political and cultural frameworks of the first-century Mediterranean
  world make it easier for Christianity to spread? \\
  Many people already worshiped cults similar to Christianity, and some believed that he
  lived by stoic principles, or that he was the embodiment of Ahura Mazda. Alexander's prior
  Hellenization allowed for Christianity to spread easily. It was also appealing to god-fearing
  non-Jews.
  \item How would you summarize Paul’s background and activities? \\
  Paul was born to a family of Pharisees. He started out his life devoted to wiping out the
  cult that worship Jesus. However, in his mid-20s, he devoted his life to spreading
  Christianity.
  \item How did Paul influence the development of early Christianity? \\
  Paul declared Judaism irrelevant to Christianity, meaning new converts would
  not have to convert to Judaism before Christianity. He also helped to spread the faith.
  \item As far as you can tell from the textbook and from the exchange between Pliny and
  Trajan, how did Roman officials think of Christians? What aspects of early Christianity
  made them suspicious? \\
  The high status of women in this religion was unsual to the Romans, and the worship of
  a criminal was worrysome. The secret nighttime meetings made Roman suspicious, and
  some thought Christianity involved cannibalism.
\end{enumerate}
\subsubsection*{The Gospel According to Matthew, Chapter 5}
\textbf{Genre}
Religious text \\
\textbf{Location of Source}
10.3, Handout (also on Drive) \\
\textbf{Content}
The author sets out a narrative of Jesus’ preaching on one occasion, focusing on the Law. \\
\textbf{Dates}
c. 30 C.E. (approximate year of Jesus’ death). Written: c. 80-85 C.E. \\
\textbf{Context}
Be aware of currents within first-century Judaism, including varied interpretations of what
the Law requires (especially Pharisee and Sadducee approaches) and apocalyptic thinking. \\
Note that the Gospel of Matthew is considered by many scholars to be the most “Jewish”
of the Gospels. To quote religious scholar Bart Ehrman, “This Gospel is distinguished by its
stress both on the Jewish-ness of Jesus and on his opposition to Judaism as he found it…
[Jesus] gathered Jewish disciples and taught them that they had to follow the Jewish Law.
At the same time, Jesus opposed the Jewish teachers of his day and condemned the way
they practiced their religion.” \\
\textbf{Author}
The author is anonymous. (“Matthew” refers to a disciple of Jesus but this attribution came
later, in the second century.) Internal evidence indicates that the author was an educated,
Greek-speaking Christian writing about fifty years after Christ’s death, recording and editing
traditions about Christ. \\
\textbf{Purpose}
The source was designed to explain the author’s understanding of Jesus’ teachings and to
win converts. \\
\\
\textbf{Discussion questions}
\begin{enumerate}
  \item Based on Chapter 5 of Matthew, how would you describe the relation between the
  teachings of Jesus and the Law of Moses? \\
  The teaching of Jesus suggested that there is a definite afterlife, and Christianity is
  Judaism with stricter commandments.
\end{enumerate}
\end{document}
